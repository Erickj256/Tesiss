\documentclass[a4paper,openright,12pt]{book}
\usepackage[utf8]{inputenc}
\usepackage[spanish,es-noshorthands]{babel}
\usepackage{amsthm}
\usepackage{amsmath}
\usepackage{amsfonts}
\usepackage{amssymb}
\usepackage{tikz}
\usepackage{caption}
\usepackage{natbib}
\usetikzlibrary{arrows.meta,positioning}
\newtheorem*{Introducción}{Introducción}
\newtheorem*{Definición}{Definición}
\newtheorem*{Lema}{Lema}
\newtheorem*{Teorema}{Teorema}
\usepackage{ragged2e}
\usepackage{graphicx}
\usepackage[left=2.5cm,right=2.5cm,top=2.5cm,bottom=2.5cm]{geometry}
\setlength{\parindent}{0mm}

\begin{document}

\section*{\Large{Introducción\\}}

Las enfermedades infecciosas son tan antiguas como los seres humanos y el resto de las especies que habitan en la tierra, estos mismos individuos han padecido numerosas enfermedades a lo largo de toda su vida. Estas enfermedades infecciosas son el resultado de la interacción entre las especies y los microorganismos que han establecido un equilibrio ecológico entre las morbilidades y mortalidades. Sin embargo, cuando los factores ambientales se ven perturbados dando paso a que exista un cambio en este equilibrio y provocando una mayor transmisión y mortalidad es cuando se establece que una enfermedad infecciosa es una epidemia. \\

Cuando el ser humano comenzó a establecer comunidades permitió que la transmisión de las enfermedades fuera en aumento dando paso a que alguna de estas se convirtiera en epidemia, como lo fue la Peste bubónica (1346 - 1353 ) en Europa  teniendo una duración aproximada de 7 años y un estimado de 75 a 200 millones de defunciones, La gripe española (1918 - 1919) que afecto a la población mundial y tuvo una duración de un año con un estimado de 50 a 100 millones de defunciones y por último la pandemia provocada por el virus del SARS-CoV-2, esta epidemia fue detectada a finales del año 2019 en Wuhan (China) la cual afecto a la población mundial, hasta el momento se estima que hay 6.7 millones de defunciones.\\

Es por esto que se tuvo la necesidad de plantear un modelo matemático que nos permitiera conocer el desarrollo de la epidemia. En el año 1927 W.O Kermack y A.G Mckendrick propusieron el modelo SIR (Susceptible, Infectado, Recuperado) el cual es uno de los modelos más usados para conocer el compartimiento de este tipo de fenómenos, a partir de este modelo nacieron modelos mucho más complejos, como lo son el SI,SIS por mencionar algunos. \\

\end{document}